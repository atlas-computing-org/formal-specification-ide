\section*{5. PREPROCESSING}

Preprocessing consists of three steps: padding the message, \(M\) (Sec. 5.1), parsing the message into message blocks (Sec. 5.2), and setting the initial hash value, \(H^{(0)}\) (Sec. 5.3).

\subsection*{5.1 Padding the Message}

The purpose of this padding is to ensure that the padded message is a multiple of 512 or 1024 bits, depending on the algorithm. Padding can be inserted before hash computation begins on a message, or at any other time during the hash computation prior to processing the block(s) that will contain the padding.

\subsection*{5.1.1 SHA-1, SHA-224 and SHA-256}

Suppose that the length of the message, \(M\), is \(\ell\) bits. Append the bit " 1 " to the end of the message, followed by \(k\) zero bits, where \(k\) is the smallest, non-negative solution to the equation \(\ell+1+k \equiv 448 \bmod 512\). Then append the 64 -bit block that is equal to the number \(\ell\) expressed using a binary representation. For example, the (8-bit ASCII) message "abc" has length \(8 \times 3=24\), so the message is padded with a one bit, then \(448-(24+1)=423\) zero bits, and then the message length, to become the 512-bit padded message\\

\includegraphics[max width=\textwidth, center]{2025_01_07_9c328d09069db93254c0g-18}

The length of the padded message should now be a multiple of 512 bits.

\subsection*{5.2 Parsing the Message}

The message and its padding must be parsed into \(N m\)-bit blocks.

\subsection*{5.2.1 SHA-1, SHA-224 and SHA-256}

For SHA-1, SHA-224 and SHA-256, the message and its padding are parsed into \(N\) 512-bit blocks, \(M^{(1)}, M^{(2)}, \ldots, M^{(N)}\). Since the 512 bits of the input block may be expressed as sixteen 32bit words, the first 32 bits of message block \(i\) are denoted \(M_{0}^{(i)}\), the next 32 bits are \(M_{1}^{(i)}\), and so on up to \(M_{15}^{(i)}\).

\subsection*{5.3 Setting the Initial Hash Value ( \(H^{(0)}\) )}

Before hash computation begins for each of the secure hash algorithms, the initial hash value, \(H^{(0)}\), must be set. The size and number of words in \(H^{(0)}\) depends on the message digest size.

\subsection*{5.3.1 SHA-1}

For SHA-1, the initial hash value, \(H^{(0)}\), shall consist of the following five 32-bit words, in hex:

\[
\begin{aligned}
H_{0}^{(0)} & =67452301 \\
H_{1}^{(0)} & =\text { efcdab89 } \\
H_{2}^{(0)} & =98 \mathrm{badcfe} \\
H_{3}^{(0)} & =10325476 \\
H_{4}^{(0)} & =\mathrm{c} 3 \mathrm{~d} 2 \mathrm{e} 1 \mathrm{f0}
\end{aligned}
\]

\section*{6. SECURE HASH ALGORITHMS}

In the following sections, the hash algorithms are not described in ascending order of size. SHA256 is described before SHA-224 because the specification for SHA-224 is identical to SHA256, except that different initial hash values are used, and the final hash value is truncated to 224 bits for SHA-224. The same is true for SHA-512, SHA-384, SHA-512/224 and SHA-512/256, except that the final hash value is truncated to 224 bits for SHA-512/224, 256 bits for SHA\(512 / 256\) or 384 bits for SHA-384.

For each of the secure hash algorithms, there may exist alternate computation methods that yield identical results; one example is the alternative SHA-1 computation described in Sec. 6.1.3. Such alternate methods may be implemented in conformance to this standard.

\subsection*{6.1 SHA-1}

SHA-1 may be used to hash a message, \(M\), having a length of \(\ell\) bits, where \(0 \leq \ell<2^{64}\). The algorithm uses 1) a message schedule of eighty 32 -bit words, 2) five working variables of 32 bits each, and 3) a hash value of five 32-bit words. The final result of SHA-1 is a 160 -bit message digest.

The words of the message schedule are labeled \(W_{0}, W_{1}, \ldots, W_{79}\). The five working variables are labeled \(\boldsymbol{a}, \boldsymbol{b}, \boldsymbol{c}, \boldsymbol{d}\), and \(\boldsymbol{e}\). The words of the hash value are labeled \(H_{0}^{(i)}, H_{1}^{(i)}, \ldots, H_{4}^{(i)}\), which will hold the initial hash value, \(H^{(0)}\), replaced by each successive intermediate hash value (after each message block is processed), \(H^{(i)}\), and ending with the final hash value, \(H^{(N)}\). SHA-1 also uses a single temporary word, \(T\).

\subsection*{6.1.1 SHA-1 Preprocessing}

\begin{enumerate}
  \item Set the initial hash value, \(H^{(0)}\), as specified in Sec. 5.3.1.
  \item The message is padded and parsed as specified in Section 5.
\end{enumerate}

\subsection*{6.1.2 SHA-1 Hash Computation}

The SHA-1 hash computation uses functions and constants previously defined in Sec. 4.1.1 and Sec. 4.2.1, respectively. Addition (+) is performed modulo \(2^{32}\).

Each message block, \(M^{(1)}, M^{(2)}, \ldots, M^{(N)}\), is processed in order, using the following steps:

For \(i=1\) to \(N\) :\\
\{
\begin{enumerate}
  \item Prepare the message schedule, \(\left\{W_{t}\right\}\) :
\[
W_{t}= \begin{cases}M_{t}^{(i)} & 0 \leq t \leq 15 \\ R O T L^{1}\left(W_{t-3} \oplus W_{t-8} \oplus W_{t-14} \oplus W_{t-16}\right) & 16 \leq t \leq 79\end{cases}
\]
  \item Initialize the five working variables, \(\boldsymbol{a}, \boldsymbol{b}, \boldsymbol{c}, \boldsymbol{d}\), and \(\boldsymbol{e}\), with the \((i-1)^{\text {st }}\) hash value:
\[
\begin{aligned}
& a=H_{0}^{(i-1)} \\
& b=H_{1}^{(i-1)} \\
& c=H_{2}^{(i-1)} \\
& d=H_{3}^{(i-1)} \\
& e=H_{4}^{(i-1)}
\end{aligned}
\]
  \item For \(t=0\) to 79 :\\
\{
\[
\begin{aligned}
& T=R O T L^{5}(a)+f_{t}(b, c, d)+e+K_{t}+W_{t} \\
& e=d \\
& d=c \\
& c=R O T L^{30}(b) \\
& b=a \\
& a=T
\end{aligned}
\]
\}\\
  \item Compute the \(i^{\text {th }}\) intermediate hash value \(H^{(i)}\) :
\[
\begin{aligned}
& H_{0}^{(i)}=a+H_{0}^{(i-1)} \\
& H_{1}^{(i)}=b+H_{1}^{(i-1)} \\
& H_{2}^{(i)}=c+H_{2}^{(i-1)} \\
& H_{3}^{(i)}=d+H_{3}^{(i-1)} \\
& H_{4}^{(i)}=e+H_{4}^{(i-1)}
\end{aligned}
\]
\}\\

After repeating steps one through four a total of \(N\) times (i.e., after processing \(M^{(N)}\) ), the resulting 160-bit message digest of the message, \(M\), is

\[
H_{0}^{(N)}\left\|H_{1}^{(N)}\right\| H_{2}^{(N)}\left\|H_{3}^{(N)}\right\| H_{4}^{(N)}
\]
