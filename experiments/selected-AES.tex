\section*{5. Algorithm Specifications}

The general function for executing AES-128, AES-192, or AES-256 is denoted by CIPHER(); its inverse is denoted by INVCIPHER().\footnotetext{Informally, these functions are sometimes called "encryption" and "decryption," but neutral terminology is appropriate because there are other applications of block ciphers besides encryption.}

The core of the algorithms for CIPHER() and INVCIPHER() is a sequence of fixed transformations of the state called a round. Each round requires an additional input called the round key; the round key is a block that is usually represented as a sequence of four words (i.e., 16 bytes).

An expansion routine, denoted by KEYEXPANSION(), takes the block cipher key as input and generates the round keys as output. In particular, the input to KEyEXPAnsion() is represented as an array of words, denoted by key, and the output is an expanded array of words, denoted by \(w\), called the key schedule.\\

The block ciphers AES-128, AES-192, and AES-256 differ in three respects: 1) the length of the key; 2) the number of rounds, which determines the size of the required key schedule; and 3) the specification of the recursion within KEyEXPANSION(). For each algorithm, the number of rounds is denoted by \(N r\), and the number of words of the key is denoted by \(N k\). (The number of words in the state is denoted by \(N b\) for Rijndael in general; in this Standard, \(N b=4\).) The specific values of \(N k, N b\), and \(N r\) are given in Table 3. No other configurations of Rijndael conform to this Standard.

For implementation issues relating to the key length, block size, and number of rounds, see Section 6.3.

Table 3. Key-Block-Round Combinations

\begin{center}
\begin{tabular}{|c|c|c|c|c|c|}
\hline
 & \multicolumn{2}{|l|}{\begin{tabular}{l}
Key length \\
\(N k\) (in bits) \\
\end{tabular}} &  & \begin{tabular}{l}
ck size \\
(in bits) \\
\end{tabular} & Number of rounds Nr \\
\hline
AES-128 & 4 & 128 & 4 & 128 & 10 \\
\hline
AES-192 & 6 & 192 & 4 & 128 & 12 \\
\hline
AES-256 & 8 & 256 & 4 & 128 & 14 \\
\hline
\end{tabular}
\end{center}

The three inputs to \(\operatorname{CIPHER}()\) are: 1) the data input in, which is a block represented as a linear array of 16 bytes; 2) the number of rounds Nr for the instance; and 3) the round keys. Thus,

\[
\begin{aligned}
& \operatorname{AES}-128(\text { in }, \text { key })=\operatorname{CiPHER}(i n, 10, \operatorname{KEYEXPANSION~}(\text { key })) \\
& \operatorname{AES}-192(\text { in }, \text { key })=\operatorname{CiPHER}(\text { in }, 12, \operatorname{KEYEXPANSION}(\text { key })) \\
& \operatorname{AES}-256(\text { in }, \text { key })=\operatorname{CIPHER}(i n, 14, \operatorname{KEYEXPANSION~}(\text { key }))
\end{aligned}
\]

The inverse permutations are defined by replacing CIPHER() with InvCipHER() in Eq. 5.1.

The specifications of \(\operatorname{CiPhER}(), \operatorname{KeyExPansion(})\), and InvCiPher() are given in Sections 5.1, 5.2 , and 5.3 , respectively.

\subsection*{5.1 Cipher()}

The rounds in the specification of \(\operatorname{CIPHER}()\) are composed of the following four byte-oriented transformations on the state:

\begin{itemize}
  \item SubBytes() applies a substitution table (S-box) to each byte.
  \item ShiftRows() shifts rows of the state array by different offsets.
  \item MixColumnS() mixes the data within each column of the state array.
  \item AddRoundKey() combines a round key with the state.
\end{itemize}

The four transformations are specified in Sections 5.1.1-5.1.4. In those specifications, the transformed bit, byte, or block is denoted by appending the symbol ' as a superscript on the original variable (i.e., by \(b_{i}^{\prime}, b^{\prime}, s_{i, j}^{\prime}\), or \(s^{\prime}\) ).\\

The round keys for \(\operatorname{AdDRoundKEY}()\) are generated by KEyExPAnsion(), which is specified in Section 5.2. In particular, the key schedule is represented as an array \(w\) of \(4 *(N r+1)\) words.

CIPHER() is specified in the pseudocode in Alg. 1.

\begin{verbatim}
Algorithm 1 Pseudocode for CIPHER()
    procedure CIPHER(in, \(N r, w\) )
        state \(\leftarrow\) in \(\triangleright\) See Sec. 3.4
        state \(\leftarrow \operatorname{ADDRoundKEY}(\) state,\(w[0.3]) \quad \triangleright\) See Sec. 5.1.4
        for round from 1 to \(\mathrm{Nr}-1\) do
            state \(\leftarrow\) SubB YtEs(state) \(\triangleright\) See Sec.5.1.1
            state \(\leftarrow\) ShIFTRows(state) \(\triangleright\) See Sec.5.1.2
            state \(\leftarrow\) MixColumns(state) \(\triangleright\) See Sec.5.1.3
            state \(\leftarrow \operatorname{ADDROUNDKEY}(\) state,\(w[4 *\) round. \(.4 *\) round +3\(])\)
        end for
        state \(\leftarrow\) SUBB YTES(state)
        state \(\leftarrow\) SHIFTRows(state)
        state \(\leftarrow \operatorname{ADDRoundKEY}(\) state,\(w[4 * N r . .4 * N r+3])\)
        return state \(\triangleright\) See Sec.3.4
    end procedure
\end{verbatim}

The first step (Line 2) is to copy the input into the state array using the conventions from Sec. 3.4. After an initial round key addition (Line 3), the state array is transformed by Nr applications of the round function (Lines 4-12); the final round (Lines 10-12) differs in that the MixColumns() transformation is omitted. The final state is then returned as the output (Line 13), as described in Section 3.4.

\subsection*{5.1.1 SubBytes()}

SUBBYTES() is an invertible, non-linear transformation of the state in which a substitution table, called an S-box, is applied independently to each byte in the state. The AES S-box is denoted by SBox().

Let \(b\) denote an input byte to SBox(), and let \(c\) denote the constant byte \(\{01100011\}\). The output byte \(b^{\prime}=\operatorname{SBOX}(b)\) is constructed by composing the following two transformations:

\begin{enumerate}
  \item Define an intermediate value \(\tilde{b}\), as follows, where \(b^{-1}\) is the multiplicative inverse of \(b\), as described in Section 4.4:
\end{enumerate}

\[
\tilde{b}= \begin{cases}\{00\} & \text { if } b=\{00\} \\ b^{-1} & \text { if } b \neq\{00\}\end{cases}
\]

\begin{enumerate}
  \setcounter{enumi}{1}
  \item Apply the following affine transformation of the bits of \(\tilde{b}\) to produce the bits of \(b^{\prime}\) :
\end{enumerate}

\[
b_{i}^{\prime}=\tilde{b}_{i} \oplus \tilde{b}_{(i+4) \bmod 8} \oplus \tilde{b}_{(i+5) \bmod 8} \oplus \tilde{b}_{(i+6) \bmod 8} \oplus \tilde{b}_{(i+7) \bmod 8} \oplus c_{i}
\]

The matrix form of Eq. (5.3) is given by Eq. (5.4) below:

\[
\left[\begin{array}{l}
b_{0}^{\prime} \\
b_{1}^{\prime} \\
b_{2}^{\prime} \\
b_{3}^{\prime} \\
b_{4}^{\prime} \\
b_{5}^{\prime} \\
b_{6}^{\prime} \\
b_{7}^{\prime}
\end{array}\right]=\left[\begin{array}{llllllll}
1 & 0 & 0 & 0 & 1 & 1 & 1 & 1 \\
1 & 1 & 0 & 0 & 0 & 1 & 1 & 1 \\
1 & 1 & 1 & 0 & 0 & 0 & 1 & 1 \\
1 & 1 & 1 & 1 & 0 & 0 & 0 & 1 \\
1 & 1 & 1 & 1 & 1 & 0 & 0 & 0 \\
0 & 1 & 1 & 1 & 1 & 1 & 0 & 0 \\
0 & 0 & 1 & 1 & 1 & 1 & 1 & 0 \\
0 & 0 & 0 & 1 & 1 & 1 & 1 & 1
\end{array}\right]\left[\begin{array}{c}
\tilde{b}_{0} \\
\tilde{b}_{1} \\
\tilde{b}_{2} \\
\tilde{b}_{3} \\
\tilde{b}_{4} \\
\tilde{b}_{5} \\
\tilde{b}_{6} \\
\tilde{b}_{7}
\end{array}\right]+\left[\begin{array}{l}
1 \\
1 \\
0 \\
0 \\
0 \\
1 \\
1 \\
0
\end{array}\right] .
\]

Figure 2 illustrates how SUBB YtES() transforms the state.\\
\includegraphics[max width=\textwidth, center]{2024_12_02_e6e04e495d18277bcb5dg-21}

Figure 2. Illustration of SUbBYteS()

The AES S-box is presented in hexadecimal form in Table 4. For example, if \(s_{r, c}=\{53\}\), then

Table 4. SBox(): substitution values for the byte xy (in hexadecimal format)

\begin{center}
\begin{tabular}{|c|c|c|c|c|c|c|c|c|c|c|c|c|c|c|c|c|c|}
\hline
 &  & \multicolumn{16}{|c|}{Y} \\
\hline
 &  & 0 & 1 & 2 & 3 & 4 & 5 & 6 & 7 & 8 & 9 & a & b & C & d & e & f \\
\hline
 & 0 & 63 & 7 c & 77 & 7b & f2 & 6b & 6 f & c5 & 30 & 01 & 67 & 2b & fe & d7 & ab & 76 \\
\hline
 & 1 & ca & 82 & c9 & \(7 d\) & fa & 59 & 47 & f0 & ad & d4 & a2 & af & 9 c & a 4 & 72 & c0 \\
\hline
 & 2 & b7 & fd & 93 & 26 & 36 & 3f & f7 & C C & 34 & a 5 & e5 & f1 & 71 & d8 & 31 & 15 \\
\hline
 & 3 & 04 & C 7 & 23 & c3 & 18 & 96 & 05 & 9a & 07 & 12 & 80 & e2 & eb & 27 & b2 & 75 \\
\hline
 & 4 & 09 & 83 & 2 C & \(1 a\) & 1 b & 6 e & 5 a & a 0 & 52 & 3b & d6 & b3 & 29 & e3 & 2 f & 84 \\
\hline
 & 5 & 53 & d1 & 00 & ed & 20 & fc & b1 & 5.b & 6a & cb & be & 39 & 4 a & 4 C & 58 & cf \\
\hline
 & 6 & d0 & ef & aa & fb & 43 & 4 d & 33 & 85 & 45 & f9 & 02 & 7 f & 50 & 3 C & 9 f & a 8 \\
\hline
x & 7 & 51 & a3 & 40 & 8 f & 92 & 9d & 38 & f5 & boc & b6 & da & 21 & 10 & ff & f3 & d2 \\
\hline
x & 8 & cd & 0 c & 13 & ec & 5 f & 97 & 44 & 17 & c4 & a 7 & 7 e & \(3 d\) & 64 & \(5 d\) & 19 & 73 \\
\hline
 & 9 & 60 & 81 & 4 f & dc & 22 & \(2 a\) & 90 & 88 & 46 & ee & b8 & 14 & de & 5 e & 0 b & db \\
\hline
 & a & e 0 & 32 & \(3 a\) & 0 a & 49 & 06 & 24 & 5 c & c2 & d3 & ac & 62 & 91 & 95 & e4 & 79 \\
\hline
 & b & e7 & C8 & 37 & 6d & 8d & d5 & 4 e & a 9 & 6 C & 56 & f4 & ea & 65 & \(7 a\) & ae & 08 \\
\hline
 & C & ba & 78 & 25 & 2 e & 1 C & a 6 & b4 & c6 & e 8 & dd & 74 & 1 f & 4b & bd & 8b & 8 a \\
\hline
 & d & 70 & 3 e & b5 & 66 & 48 & 03 & f6 & 0 e & 61 & 35 & 57 & b9 & 86 & c1 & \(1 d\) & 9 e \\
\hline
 & e & e1 & f8 & 98 & 11 & 69 & d9 & 8 e & 94 & 9b & 1 1e & 87 & e 9 & ce & 55 & 28 & df \\
\hline
 & f & 8 c & a1 & 89 & 0d & bf & e6 & 42 & 68 & 41 & 99 & 2d & 0 f & b0 & 54 & bb & 16 \\
\hline
\end{tabular}
\end{center}

the substitution value would be determined by the intersection of the row with index ' 5 ' and the column with index ' 3 ' in Table 4, so that \(s_{r, c}^{\prime}=\{\mathrm{ed}\}\).

\subsection*{5.1.2 ShIFTRows()}
SHIFTRows() is a transformation of the state in which the bytes in the last three rows of the state are cyclically shifted. The number of positions by which the bytes are shifted depends on the row index \(r\), as follows:

\[
s_{r, c}^{\prime}=s_{r,(c+r) \bmod 4} \quad \text { for } 0 \leq r<4 \text { and } 0 \leq c<4
\]

SHIFTRows() is illustrated in Figure 3. In that representation of the state, the effect is to move each byte by \(r\) positions to the left in the row, cycling the left-most \(r\) bytes around to the right end of the row. The first row, where \(r=0\), is unchanged.\\
\includegraphics[max width=\textwidth, center]{2024_12_02_e6e04e495d18277bcb5dg-23(1)}

\begin{center}
\begin{tabular}{|l|l|l|l|}
\hline
\multicolumn{4}{|c|}{\(s\)} \\
\hline
\(s_{0,0}\) & \(s_{0,1}\) & \(s_{0,2}\) & \(s_{0,3}\) \\
\hline
\(s_{1,0}\) & \(s_{1,1}\) & \(s_{1,2}\) & \(s_{1,3}\) \\
\hline
\(s_{2,0}\) & \(s_{2,1}\) & \(s_{2,2}\) & \(s_{2,3}\) \\
\hline
\(s_{3,0}\) & \(s_{3,1}\) & \(s_{3,2}\) & \(s_{3,3}\) \\
\hline
\end{tabular}
\end{center}

\begin{center}
\includegraphics[max width=\textwidth]{2024_12_02_e6e04e495d18277bcb5dg-23}
\end{center}

Figure 3. Illustration of SHIFTRows()

\subsection*{5.1.3 MixColumns()}
MixColumns( ) is a transformation of the state that multiplies each of the four columns of the state by a single fixed matrix, as described in Section 4.3, with its entries taken from the following word:

\[
\left[a_{0}, a_{1}, a_{2}, a_{3}\right]=[\{02\},\{01\},\{01\},\{03\}] .
\]

Thus,

\[
\left[\begin{array}{l}
s_{0, c}^{\prime} \\
s_{1, c}^{\prime} \\
s_{2, c}^{\prime} \\
s_{3, c}^{\prime}
\end{array}\right]=\left[\begin{array}{llll}
02 & 03 & 01 & 01 \\
01 & 02 & 03 & 01 \\
01 & 01 & 02 & 03 \\
03 & 01 & 01 & 02
\end{array}\right]\left[\begin{array}{l}
s_{0, c} \\
s_{1, c} \\
s_{2, c} \\
s_{3, c}
\end{array}\right] \quad \text { for } 0 \leq c<4
\]

so that the individual output bytes are defined as follows:

\[
\begin{aligned}
s_{0, c}^{\prime} & =\left(\{02\} \bullet s_{0, c}\right) \oplus\left(\{03\} \bullet s_{1, c}\right) \oplus s_{2, c} \oplus s_{3, c} \\
s_{1, c}^{\prime} & =s_{0, c} \oplus\left(\{02\} \bullet s_{1, c}\right) \oplus\left(\{03\} \bullet s_{2, c}\right) \oplus s_{3, c} \\
s_{2, c}^{\prime} & =s_{0, c} \oplus s_{1, c} \oplus\left(\{02\} \bullet s_{2, c}\right) \oplus\left(\{03\} \bullet s_{3, c}\right) \\
s_{3, c}^{\prime} & =\left(\{03\} \bullet s_{0, c}\right) \oplus s_{1, c} \oplus s_{2, c} \oplus\left(\{02\} \bullet s_{3, c}\right) .
\end{aligned}
\]

Figure 4 illustrates MixColumns().\\
\includegraphics[max width=\textwidth, center]{2024_12_02_e6e04e495d18277bcb5dg-24(1)}

Figure 4. Illustration of MixColumns()

\subsection*{5.1.4 AddRoundKey()}
ADDRoundKEY() is a transformation of the state in which a round key is combined with the state by applying the bitwise XOR operation. In particular, each round key consists of four words from the key schedule (described in Section 5.2), each of which is combined with a column of the state as follows:

\[
\left[s_{0, c}^{\prime}, s_{1, c}^{\prime}, s_{2, c}^{\prime}, s_{3, c}^{\prime}\right]=\left[s_{0, c}, s_{1, c}, s_{2, c}, s_{3, c}\right] \oplus\left[w_{(4 * \text { round }+c)}\right] \quad \text { for } 0 \leq c<4
\]

where round is a value in the range \(0 \leq\) round \(\leq N r\), and \(w[i]\) is the array of key schedule words described in Section 5.2. In the specification of Cipher(), AddRoundKey() is invoked \(N r+1\) times - once prior to the first application of the round function (see Alg. 1) and once within each of the \(N r\) rounds, when \(1 \leq\) round \(\leq N r\).

The action of this transformation is illustrated in Fig. 5, where \(l=4 *\) round. The byte address within words of the key schedule was described in Sec. 3.5.\\
\includegraphics[max width=\textwidth, center]{2024_12_02_e6e04e495d18277bcb5dg-24}

Figure 5. Illustration of AddRoundKEy()

\begin{verbatim}
Algorithm 2 Pseudocode for KEYEXPANSION()
    procedure KEYEXPANSION(key)
        \(i \leftarrow 0\)
        while \(i \leq N k-1\) do
            \(w[i] \leftarrow k e y[4 * i . .4 * i+3]\)
            \(i \leftarrow i+1\)
        end while \(\quad \triangleright\) When the loop concludes, \(i=N k\).
        while \(i \leq 4 * N r+3\) do
            temp \(\leftarrow w[i-1]\)
            if \(i \bmod N k=0\) then
                temp \(\leftarrow \operatorname{SubWORd}(\operatorname{RotWORD}(\) temp \()) \oplus \operatorname{Rcon}[i / N k]\)
            else if \(N k>6\) and \(i \bmod N k=4\) then
                temp \(\leftarrow \operatorname{SUBWORD}(\) temp \()\)
            end if
            \(w[i] \leftarrow w[i-N k] \oplus\) temp
            \(i \leftarrow i+1\)
        end while
        return \(w\)
    end procedure
\end{verbatim}

Figures 6, 7, and 8 illustrate KEYEXPANSION() for AES-128, AES-192, and AES-256.

\subsection*{5.3 INVCIPHER()}
To implement InvCIPHER(), the transformations in the specification of CIPHER() (Section 5.1) are inverted and executed in reverse order. The inverted transformations of the state - denoted by InvShiftRows(), InvSubB ytes(), InvMixColumns(), and AddRoundKey() - are described in Sections 5.3.1-5.3.4.

InvCipHER() is described in the pseudocode in Alg. 3, where the array \(w\) denotes the key schedule, as described in Section 5.2.\\
\includegraphics[max width=\textwidth, center]{2024_12_02_e6e04e495d18277bcb5dg-27}

Figure 6. KEyExpansion() of AES-128 to generate the words \(w[i]\) for \(4 \leq i<44\), where \(l\) ranges over the multiples of 4 between 0 and 36\\
\includegraphics[max width=\textwidth, center]{2024_12_02_e6e04e495d18277bcb5dg-28}

Figure 7. KeyExpansion() of AES-192 to generate the words \(w[i]\) for \(6 \leq i<52\), where \(l\) ranges over the multiples of 6 between 0 and 42\\
\includegraphics[max width=\textwidth, center]{2024_12_02_e6e04e495d18277bcb5dg-29}

Figure 8. KEyExpansion() of AES-256 to generate the words \(w[i]\) for \(8 \leq i<60\), where \(l\) ranges over the multiples of 8 between 0 and 48

\subsection*{5.3.3 InvMixColumns()}
InvMixColumns() is the inverse of MixColumns(). In particular, InvMixColumns() multiplies each of the four columns of the state by a single fixed matrix, as described in Section 4.3, with its entries taken from the following word:

\[
\left[a_{0}, a_{1}, a_{2}, a_{3}\right]=[\{0 \mathrm{e}\},\{09\},\{0 \mathrm{~d}\},\{0 \mathrm{~b}\}] .
\]

Thus,

\[
\left[\begin{array}{c}
s_{0, c}^{\prime} \\
s_{1, c}^{\prime} \\
s_{2, c}^{\prime} \\
s_{3, c}^{\prime}
\end{array}\right]=\left[\begin{array}{cccc}
0 \mathrm{e} & 0 \mathrm{~b} & 0 \mathrm{~d} & 09 \\
09 & 0 \mathrm{e} & 0 \mathrm{~b} & 0 \mathrm{~d} \\
0 \mathrm{~d} & 09 & 0 \mathrm{e} & 0 \mathrm{~b} \\
0 \mathrm{~b} & 0 \mathrm{~d} & 09 & 0 \mathrm{e}
\end{array}\right]\left[\begin{array}{l}
s_{0, c} \\
s_{1, c} \\
s_{2, c} \\
s_{3, c}
\end{array}\right] \quad \text { for } 0 \leq c<4
\]

As a result of this matrix multiplication, the four bytes in a column are replaced by the following:

\[
\begin{aligned}
s_{0, c}^{\prime} & =\left(\{0 \mathrm{e}\} \bullet s_{0, c}\right) \oplus\left(\{0 \mathrm{~b}\} \bullet s_{1, c}\right) \oplus\left(\{0 \mathrm{~d}\} \bullet s_{2, c}\right) \oplus\left(\{09\} \bullet s_{3, c}\right) \\
s_{1, c}^{\prime} & =\left(\{09\} \bullet s_{0, c}\right) \oplus\left(\{0 \mathrm{e}\} \bullet s_{1, c}\right) \oplus\left(\{0 \mathrm{~b}\} \bullet s_{2, c}\right) \oplus\left(\{0 \mathrm{~d}\} \bullet s_{3, c}\right) \\
s_{2, c}^{\prime} & =\left(\{0 \mathrm{~d}\} \bullet s_{0, c}\right) \oplus\left(\{09\} \bullet s_{1, c}\right) \oplus\left(\{0 \mathrm{e}\} \bullet s_{2, c}\right) \oplus\left(\{0 \mathrm{~b}\} \bullet s_{3, c}\right) \\
s_{3, c}^{\prime} & =\left(\{0 \mathrm{bb}\} \bullet s_{0, c}\right) \oplus\left(\{0 \mathrm{~d}\} \bullet s_{1, c}\right) \oplus\left(\{09\} \bullet s_{2, c}\right) \oplus\left(\{0 \mathrm{e}\} \bullet s_{3, c}\right) .
\end{aligned}
\]

\subsection*{5.3.4 Inverse of AddRoundKey()}
AdDRoundKEY(), described in Section 5.1.4, is its own inverse.

\subsection*{5.3.5 EQInvCIPher()}
Several properties of the AES algorithm allow for an alternative specification of the inverse of CIPHER(), called the equivalent inverse cipher, denoted by EQINVCIPHER(). In the specification of EQInVCIPHER(), the transformations of the round function of the cipher in Alg. 1 are directly replaced by their inverses in EQInvCipher(), in the same order. The efficiency of this structure in comparison to the specification of InvCIPHER() in Alg. 3 is explained in the Rijndael proposal document [2].

The pseudocode for the equivalent inverse cipher, given in Alg. 4, uses a modified key schedule, denoted by the word array \(d w\). The routine to generate \(d w\) is an extension of KEyExpansion(), denoted by KEyEXPAnSIONEIC(), whose pseudocode is given in Alg. 5.

\begin{verbatim}
Algorithm 4 Pseudocode for EQINVCIPHER()
    procedure EQINvCIPHER( \(i n, N r, d w)\)
        state \(\leftarrow\) in
        state \(\leftarrow \operatorname{ADDRoundKEY}(\) state,\(d w[4 * N r . .4 * N r+3])\)
        for round from \(N r-1\) downto 1 do
            state \(\leftarrow \operatorname{InVSUBBYTES(state)}\)
            state \(\leftarrow \mathrm{INVSHIFTROWS(state)}\)
            state \(\leftarrow\) InVMixColumns(state)
            state \(\leftarrow \operatorname{ADDRounDKEY}(\) state,\(d w[4 *\) round.. \(4 *\) round +3\(])\)
        end for
        state \(\leftarrow \operatorname{InVSUBB}\) YTES(state)
        state \(\leftarrow\) INVSHIFTROWS(state)
        state \(\leftarrow \operatorname{ADDRoundKEY}(\) state,\(d w[0 . .3])\)
        return state
    end procedure
\end{verbatim}

\begin{verbatim}
Algorithm 5 Pseudocode for KEYEXPANSIONEIC()
    procedure KEYEXPANSIONEIC(key)
        \(i \leftarrow 0\)
        while \(i \leq N k-1\) do
            \(w[i] \leftarrow k e y[4 i . .4 i+3]\)
            \(d w[i] \leftarrow w[i]\)
            \(i \leftarrow i+1\)
        end while \(\quad \triangleright\) When the loop concludes, \(i=N k\).
        while \(i \leq 4 * N r+3\) do
            temp \(\leftarrow w[i-1]\)
            if \(i \bmod N k=0\) then
                temp \(\leftarrow \operatorname{SUBWORD}(\operatorname{RotWORD}(\) temp \()) \oplus R \operatorname{con}[i / N k]\)
            else if \(N k>6\) and \(i \bmod N k=4\) then
                temp \(\leftarrow \operatorname{SUBWORD}(\) temp \()\)
            end if
            \(w[i] \leftarrow w[i-N k] \oplus t e m p\)
            \(d w[i] \leftarrow w[i]\)
            \(i \leftarrow i+1\)
        end while
        for round from 1 to \(\mathrm{Nr}-1\) do
            \(i \leftarrow 4 *\) round
            \(d w[i . . i+3] \leftarrow \operatorname{InVMixCoLumnS}(d w[i . . i+3]) \quad \triangleright\) Note change of type.
        end for
        return \(d w\)
    end procedure
\end{verbatim}

The first and last round keys in \(d w\) are the same as in \(w\); the modification of the other round keys is described in Lines 19-22. The comment in Line 21 refers to the input to InvMixColumns(): the one-dimensional array of words is converted to a two-dimensional array of bytes, as in Fig. 1.
